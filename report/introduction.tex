\section{Introduction}
Automatic speech recognition (ASR) means translating a spoken utterance into written text. It can, for example, be used for voice assistants or automatic transcription of audio or video files. While \gls{stt} pre-trained models for English are available, German models or even Swiss German models are rare or not existing \cite{Agarwal2019GermanES}. Swiss German has a wide variety of different Swiss German dialects, with a huge difference in words, pronunciation, even to the point of sounding like a different language. Swiss German has relatively little speakers (around 5 million) and there is hardly any standardized spelling. This leads to Standard German being one of the official writing language in Switzerland. As there is no official Swiss German spelling, most speakers using written Swiss German just use their own spelling which resembles mostly a phonetical translation. This leads to huge variance within Swiss German writing \citep{pluss2020}. It makes therefore sense to translate spoken Swiss German to Standard German in order to correspond to the official language situation. Tackling a standardized translation of different spoken Swiss German dialects into standardized German text requires a vast amount of data and fine tuning. The \gls{stc} 2021 proposed a shared task to tackle this problem and provided a dataset
(\gls{stcd} to train and fine tune on. This paper shows what kind of experiments, data, and approaches the authors used to tackle this problem. The proposed task is very complex as it includes not only a \gls{stt} conversion but also translation from Swiss German to Standard German which can be refered to as speech translation \cite{pluss2020}. Additionally, it possibly includes domain shift, as the training data stems only from Swiss parliament speeches while the domain of the test set is unclear. 
This paper presents a overview on the previous research done within this area, an introduction to the DeepSpeech model used for tackling the task and our experiments and results.  
